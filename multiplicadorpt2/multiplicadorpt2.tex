\documentclass[12pt]{article}
\usepackage[utf8]{inputenc}
\usepackage[T1]{fontenc}
\usepackage{times}
\usepackage{indentfirst}
\usepackage{colortbl}
\usepackage[table]{xcolor}
\usepackage{booktabs}
\usepackage{longtable}
\usepackage{siunitx}
\usepackage{graphicx}
\usepackage{float}
\usepackage{sectsty}
\usepackage{hyperref}


\author{Lorenzo Costa Miranda}

\date{} %remove date


\usepackage{geometry}
\usepackage{setspace}

\geometry{
	top=2cm,
	bottom=2cm,
	left=3cm,
	right=2cm}

\setlength{\parskip}{1.5em}
\setlength{\parindent}{0pt}

\singlespacing

\setlength{\parindent}{2em}
\setlength{\parskip}{0,5em} % Define o espaço entre parágrafos
\renewcommand{\baselinestretch}{1.5}

\usepackage{titlesec}

\begin{document}

\section*{Oferta de Moeda no Brasil entre 2003 à 2023}
\hspace{4.5cm} LORENZO COSTA MIRANDA
\vspace*{5pt}

As estatísticas dos agregados monetários são extremamente importantes para compreender o comportamento e as preferências dos indivíduos quanto à moeda. Sucintamente, há quatro principais grupos de ativos financeiros na ordem de liquidez: M1, M2, M3 e M4. M1 corresponde ao grupo mais líquido da economia e é constituído pelo Papel Moeda no Poder do Público e os depósitos à vista. M2 é relativamente menos líquido, e por isso, tanto o custo de transação, quanto o tempo para se converter em moeda é superior. Ele é composto por M1 mais depósitos especiais remunerados, depósitos de poupança e títulos emitidos por instituições depositárias. M3 é grupo menos flexível, e é o conjunto de M2 mais quotas de renda fica e operações compromissadas registradas no SELIC. M4 é o grupo menos líquido da economia formado por M3 e títulos públicos de alta liquidez. Nos meses de 2003 à 2023 no Brasil, M1, M2, M3 e M4 atingiram em seus ápices, respectivamente, os valores de 653.420.212; 5.913.356.669; 10.857.133.543 e 11.948.284.621. Já os seus mínimos corresponderam a 86.231.763; 385.467.653; 696.054.502 e 810.272.451. Em termos médios, pôde-se atestar os valores aproximados e respectivos de 311.848.406; 2.066.000.000; 4.239.000.000 e 4.616.000.000.

Em uma análise particular, para conceber as estatísticas no momento de instabilidades política e econômica verificadas no período de impeachment, foi averiguado o tempo de um ano, de junho de 2015 até maio de 2016, de um pouco antes do início do crescimento inflacionário e principalmente da SELIC até o mês de início do processo de afastamento do cargo da presidente. Nesse período, observa-se que o máximo das agregados monetários foram respectivamente: 347.220.589 (12-2015); 2.224.143.094 (12-2015); 5.008.017.964 (05-2016) e 5.453.699.716 (05-2016). O IPCA inaugurou seu crescimento a partir de setembro de 2015, tendo algumas quedas em novembro e dezembro, mas retomou seu aumento e finalmente atingiu seu valor máximo do período de 1.27(\%) em janeiro de 2016, em contraposição com a média de IPCA dos 20 anos de 0.450, e, partir daí, passou a decrescer de forma significativa, apresentando depois um aumento de março à maio. A taxa SELIC atingiu seu valor máximo de 14.25 já em agosto de 2015, uma taxa elevada comparada à média dos 20 anos de 11.44, e a partir daí permaneceu constante até maio de 2016. Quando a taxa SELIC apresentou aumento, em questão de 2 meses, os quatro grupos manifestaram crescimento em porcentagens parecidas. Em novembro, houve grande aumento de M1 e M4, mas depois de dezembro, mês de ápice da taxa de IPCA, todos os agregados monetários passam a regredir, porém, M1, principalmente, e M2 caem em magnitudes consideravelmente, bem maiores que de M3 e M4.

\end{document}