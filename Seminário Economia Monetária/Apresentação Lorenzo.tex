\documentclass[10pt]{Beamer}
\usepackage{fancybox}
\usepackage[most]{tcolorbox}
\usepackage{tikz}
\usepackage[utf8]{inputenc}
\usepackage{graphicx}



\usetheme{CambridgeUS}
\usecolortheme{sidebartab}
\usefonttheme{serif}


\title [Apresentação]{Apresentação: Uma Breve Revisão do Novo Consenso Macroeconômico a Partir da Crise de 2007/08: Questionamentos internos acerca da condução da política econômica pós-crise}

\author{Lorenzo Costa Miranda}

\institute{UFT Palmas}

\date{\today}

\begin{document}

\frame{\titlepage}
\frame{\tableofcontents}

\section{Objetivo do Artigo}
\frame{\frametitle{Objetivos do Artigo}


\begin{itemize}

\begin{block}
	

\item Analisar os debates quanto à revisão pós-crise do \textit{subprime} no Novo Consenso Macroeconômico. 

\end{block}	

\begin{block}

\item Evidenciar as controvérsias quanto à estabilidade econômica supostamente gerada pelo Novo Consenso Macroeconômico.

\end{block}
\end{itemize}
}

\section{Introdução}
\frame{\frametitle{Introdução}

\begin{itemize}
	
\item Contexto Histórico



\item O que é o Novo Consenso Macroeconômico (NCM)


\item Implantação e Consolidação do Regime de Metas de Inflação (RMI)



\begin{tcolorbox}[drop fuzzy shadow=ShadowColor]
Portanto, separa-se duas principais hipóteses na reavaliação das teorias:

-As mudanças na condução da política econômica são marginais, não afetando a essência do NCM. 
 
-As políticas monetárias não convencionais são consideradas instrumentos para períodos de ruptura econômica, mas não para períodos nos quais a economia opera de forma eficiente.
\end{tcolorbox}


\end{itemize}

}

\section{Métodos e Arcabouço Teórico}
\frame{\frametitle{Métodos e Arcabouço Teórico}
	
\begin{itemize}
	
\item Busca-se abordar a base teórica do NCM e demonstrar as contraposições de diversos autores como uma forma de debate. 

\vskip0,5cm

\item Estabelecer uma análise comparativa.

\vskip0,5cm
 
\item Ressaltar a quebra de paradigmas baseada nos fenômenos resultantes da recessão econômica.

  
\end{itemize}

}

\frame{\frametitle{Métodos e Arcabouço Teórico}
	
	
	
\begin{itemize}
\begin{block} 
		
		
\item Novos Clássicos $\rightarrowtail$ Curva de Lucas $\rightarrowtail$ Inconsistência temporal da PM discricionária $\rightarrowtail$ BC independente $\rightarrowtail$ RMI

\vskip0,5cm

\item Novos Keynesianos $\rightarrowtail$ Curva de Phillips com rigidez de preços e IS "foward looking" $\rightarrowtail$ RMI

\vskip0,5cm

\item Ciclo Reais de Negócios $\rightarrowtail$ Regra de Taylor $\rightarrowtail$ Ancora nominal (inflação) e instrumento de política monetária (taxa de juros e comunicação) $\rightarrowtail$ RMI
		
\end{block}
\end{itemize}	
 
}

\frame{
\frametitle{Métodos e Arcabouço Teórico}

Os 8 princípios do NCM
\begin{itemize}
\begin{block} 


\item A inflação é sempre e em toda parte um fundamento monetário.
\item A estabilidade de preços traz benefícios importantes.
\item Não existe compromisso de longo prazo entre desemprego e inflação.
\item As expectativas desempenham um papel crucial na determinação da inflação e na transmissão da política monetária à macroeconomia. 
\item As taxas de juros reais precisam subir com o aumento da inflação, ou seja, o princípio de Taylor.
	
\end{block}
\end{itemize}	

}


\frame{
\frametitle{Métodos e Arcabouço Teórico}

\begin{itemize}
	\begin{block} 
		
		
\item A política monetária está sujeita ao problema da inconsistência temporal.
\item A independência do Banco central ajuda a melhorar a eficiência da política monetária.
\item O compromisso com uma âncora nominal forte é fundamental para produzir bons resultados de política monetária. 
	
	\end{block}
\end{itemize}	
}

\section{Contribuição do Artigo}
\frame{
\frametitle{Contribuição do Artigo}
\\

\vskip0,5cm

\shadowbox{Política Monetária}

\begin{minipage}[t][\textheight][t]{\textwidth}
	\vspace*{0pt} % Espaço vertical no topo
	\begin{tcolorbox}[colback=gray!20, colframe=gray!50, title=, sharp corners]

\begin{itemize}
\item Metas de inflação baixas são danosas à efetividade da política monetária anticíclica.


\item Crítica quanto a utilização de um único índice para sugerir a inflação quando este é comparado com o produto ou o preço dos ativos.


\item Aceitava-se a \textit{doutrina greenspan} no que tange a conduta da política monetária. Mishkin propõe a utilização de políticas macroprudenciais e Woodford sugere uma nova meta ao RMI.
  

\end{itemize}	

	\end{tcolorbox}
\end{minipage}

}

\frame{
\frametitle{Contribuição do Artigo}

\begin{tcolorbox}[drop fuzzy shadow=ShadowColor]
	
Taylor, entretanto, discorda tanto da regulamentação macroprudencial, quanto ao gerenciamento do risco. Ele defende que a política monetária deveria continuar usando a regra de condução de Taylor, como foi durante a década de 1990, que garantiu a estabilidade do nível de preços, do produto e das condições financeiras por todo esse tempo. Porém, no pré-crise de 2008 a taxa de juros de curto prazo estava extremamente baixa, ampliando o crédito ao setor imobiliário e gerando mais risco.

\end{tcolorbox}


\begin{minipage}[t][\textheight][t]{\textwidth}
\vspace*{0pt} % Espaço vertical no topo
\begin{tcolorbox}[colback=gray!20, colframe=gray!50, title=, sharp corners]
		
\begin{itemize}
\item Política de sinalização, ou gestão de expectativas, proposta por Eggertsson e Woodford.
\item Comunicação do Banco central com o público.
			
\end{itemize}	
		
\end{tcolorbox}
\end{minipage}

}

\frame{
\frametitle{Contribuição do Artigo}

\vskip0.5cm

\shadowbox{Política Financeira}

\begin{minipage}[t][\textheight][t]{\textwidth}
\vspace*{0pt} % Espaço vertical no topo

\begin{tcolorbox}[drop fuzzy shadow=ShadowColor]
Desde 1970 passou a ser deixada de lado em relação à política monetária, Segundo a irrelevância teórica da intermediação financeira. Sofreu um certo abandono da supervisão como uma ferramenta macroeconômica. 
\end{tcolorbox}


\begin{tcolorbox}[colback=gray!20, colframe=gray!50, title=, sharp corners]
		
\begin{itemize}
\item Inicialmente foi atribuída o \textit{princípio Tinbergen} de modo que a política financeira foi substituída pela regulamentação microprudencial das instituições depositárias. Dessa forma, mostrou-se vulnerabilidade da política monetária no pré-crise e não houve redução de risco. 
\item Mishkin contrapõe a \textit{doutrina Greenspan} e afirma que a regulamentação microprudencial não é ideal para as situações de falha de mercado. Blanchard também critica essa ponto.

\end{itemize}	
\end{tcolorbox}
\end{minipage}

}

\frame{
\frametitle{Contribuição do Artigo}


\begin{minipage}[t][\textheight][t]{\textwidth}
	\vspace*{0pt} % Espaço vertical no topo
\begin{tcolorbox}[colback=gray!20, colframe=gray!50, title=, sharp corners]
	
\begin{itemize}

\item Quebra da dicotomia entre a política monetária e a política financeira. 
\item Rejeita-se o princípio de Tinbergen. Coordenação entre os instrumentos monetários e prudenciais.
\item Taylor ainda critica a utilização de políticas regulatórias. A taxa de juros é o melhor instrumento.			
\end{itemize}	
\end{tcolorbox}
\end{minipage}

}


\frame{
\frametitle{Contribuição do Artigo}



\shadowbox{Política Fiscal}

\begin{minipage}[t][\textheight][t]{\textwidth}
\vspace*{0pt} % Espaço vertical no topo
	
\begin{tcolorbox}[drop fuzzy shadow=ShadowColor]

Inicialmente, a política fiscal foi subestimada pelas críticas dos monetaristas e novo-clássicos por causa do efeito \textit{crowding out} que reduziu a sua prioridade em detrimento da regra de controle orçamentário para dar suporte à estabilização de preços.

\end{tcolorbox}
	
	
\begin{tcolorbox}[colback=gray!20, colframe=gray!50, title=, sharp corners]
		
\begin{itemize}
\item Em crises, é um importante instrumento anticíclico. Nesses momentos, dada as expectativas baixas, pouca força crítica havia para operação da política fiscal.
\item Com os gastos, é questão de tempo para ocorrer a monetização de dívida.
\item Default da dívida sempre gera desestabilização dos preços. 
\item Mikshin afirma que só em crise a política fiscal é cogitável. Blanchard reforça o papel anticíclico dessa política.

			
\end{itemize}	
\end{tcolorbox}
\end{minipage}

}

\frame{
\frametitle{Contribuição do Artigo}
	

	
\shadowbox{Política Cambial}
	
\begin{minipage}[t][\textheight][t]{\textwidth}
\vspace*{0pt} % Espaço vertical no topo
		
\begin{tcolorbox}[drop fuzzy shadow=ShadowColor]
			
A crise resultou em grande volatilidade do fluxo de capitais nas economias, principalmente naquelas em que havia abertura da conta de capitais, atividade econômica menos diversificada e mercados financeiros mais densos, isto é, economias em desenvolvimento no geral, que acabou por desestabilizar o câmbio delas. 
			
\end{tcolorbox}
		
		
\begin{tcolorbox}[colback=gray!20, colframe=gray!50, title=, sharp corners]
			
\begin{itemize}
\item RMI como determinante da política cambial. Porém, nos países emergentes a situação é diferente. 
\item Em crises, esses países necessitam intervir no mercado cambial. Porém Ostry é contrário em prol da política monetária. 
\item Meta da taxa de câmbio real e a intervenção estatal como componentes do RMI.
\item Após crise de 2008, o FMI aceitou controles de capitais em circunstâncias específicas.
				
				
\end{itemize}	
\end{tcolorbox}
\end{minipage}
	
}


\frame{
\frametitle{Contribuição do Artigo}
	
\vskip0,5cm
	
\shadowbox{Revisão dos Modelos DSGE}
	
\begin{minipage}[t][\textheight][t]{\textwidth}
\vspace*{0pt} % Espaço vertical no topo
		
\begin{tcolorbox}[drop fuzzy shadow=ShadowColor]
			
Mikshin questiona a validade de duas hipóteses normalmente adotadas.
			
\end{tcolorbox}
		
		
\begin{tcolorbox}[colback=gray!20, colframe=gray!50, title=, sharp corners]
			
\begin{itemize}
\item Crítica à política monetária ótima, a partir da maximização de uma função linear quadrática dos desvios de inflação e do produto. 
\item A segunda questão é a necessidade dos modelos DSGE incorporarem possibilidades de fricções financeiras e abandonarem a estrutura de agente representativo.
\item Proposta de Woodford.
				
				
\end{itemize}	
\end{tcolorbox}
\end{minipage}
	
}



\end{document}